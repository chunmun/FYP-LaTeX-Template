\documentclass[fyp]{socreport}
\usepackage{fullpage}
\usepackage{parskip}
\usepackage{amsmath}
\usepackage{MnSymbol}

\usepackage[usenames,dvipsnames,svgnames,table]{xcolor}
\usepackage[english]{babel}
\usepackage[
backend=biber,
style=alphabetic,
sorting=ynt
]{biblatex}
\addbibresource{socreport.bib}

\begin{document}
\pagenumbering{roman}

\title{Hands-Free Programming - Talk to Code}
\author{Soon Chun Mun}
\projyear{2015/16}
\projnumber{H0201120}
\advisor{Prof. Ooi Wei Tsang}

\deliverables{
  \item Report: 1 Volume}
\maketitle

\newcommand{\vp}{$\vphantom{equals to}$}
\newcommand{\tagw}[3]{$\underbrace{\hbox{\vp #1}}_{\hbox{\color{#3} \vp \textbf{#2}}}$}
\newcommand{\tago}[1]{\tagw{#1}{O}{Grey}}
\newcommand{\tagfor}[1]{\tagw{#1}{for}{red}}
\newcommand{\tagfi}[1]{\tagw{#1}{iters}{red}}

\newcommand{\tagfn}[1]{\tagw{#1}{fn-name}{Blue}}

\newcommand{\tagcond}[1]{\tagw{#1}{cond}{blue}}
\newcommand{\tagif}[1]{\tagw{#1}{if}{blue}}
\newcommand{\tagvn}[1]{$\underbrace{\hbox{\vp #1}}_{\hbox{\vp \textbf{{\color{ForestGreen} var}-name}}}$}
\newcommand{\tagvv}[1]{$\underbrace{\hbox{\vp #1}}_{\hbox{\vp \textbf{{\color{ForestGreen} var}-val}}}$}
\newcommand{\tagvm}[1]{$\underbrace{\hbox{\vp #1}}_{\hbox{\vp \textbf{{\color{ForestGreen} var}-mod}}}$}
\newcommand{\tagva}[1]{$\underbrace{\hbox{\vp #1}}_{\hbox{\vp \textbf{{\color{ForestGreen} var}-ass}}}$}

\begin{abstract}

In this report, we explore a method to identify semantic roles within
transcribed text as part of a pipeline to turn spoken speech into
code. The problem involves extracting the words that form the components
of code and command by correctly labelling them in their various semantic
context.

\begin{descriptors}
    \item C5 Computer System Implementation
\end{descriptors}
\begin{keywords}
	Semantic Role Labelling, Natural Language Processing, Neural Network
\end{keywords}
\begin{implement}
	Python 3.5.1 Theano 0.7
\end{implement}
\end{abstract}

\begin{acknowledgement}
   I would like to thank my friends, families and advisors.
   Without them, I would not have be able to complete this project.
\end{acknowledgement}



\listoffigures
\listoftables
\tableofcontents

\chapter{Introduction}

\section{Background}
This project embarks on constructing a possible framework which human
programmers are able to use speech to directly write code instead of typing
them. Conversion of human speech to code requires an understanding of code
structure occur in spoken code.

The code structures that we explored involved variable declaration, loops,
if-else conditions and function declarations.

Although spoken code contains nuances of natural humans language, we discovered
that the structures are syntactically very close to actual written code. In
fact even when humans express code that is not syntactically correct, it seems
that the number of ways they are spoken is very consistent and similar.  For
example, in the construction of a \texttt{for} loop. Many may choose to
express in the following ways:

\begin{verbatim}
  1. for i equals to 0 i less than 10 i plus plus
  2. for loop that runs 10 times
  3. create a for loop inside that runs 10 times
\end{verbatim}

We see that although the first sentence is an accurate reflection of how code
may be written as it is spoken, the second version is equally likely to occur.
The second version is a lot more natural even though it fails to specify the
loop variable or invariants because that has been abstracted away when the main
purpose of the code is to run a loop a number of times.

The last example is a sentence which we extracted from the transcribed corpus
that we used with our experiments. We see that the sentence contains
additional semantic context such as an explicit creation of the loop and that
the code within the loop should run 10 times.

All 3 examples clearly expresses the idea of creating a for loop that runs a
set number of times and our project intends to finally output C code that is
very similar to the first sentence. However, there are enough differences
between them that this cannot be directly approached without considering
the robustness of the method of translation in the presence of missing
information or word paraphrasing in non-programming keywords.

This motivates us to model the problem in terms of \textbf{Semantic Role
Labelling} where semantically relevant keywords of the given sentence are
identified before further processing can infer missing information.

\section{The Problem}

The problem we tackle here involves the part of the pipeline where accurately
transcribed text is labelled with semantic tags such as \texttt{variable-name},
or \texttt{if}. The model which is adopted is a sequence prediction given
the sentence. The predicted sequence represents the sequence of semantic tags.

% TODO include diagram of tagged sentence
\begin{enumerate}
\itemsep0em
  \item \tagfor{for} \tagvn{i} \tagva{equals to} \tagvv{0} \tagvn{i} \tagcond{less than}
    \tagvv{10} \tagvn{i} \tagvm{plus plus}
  \item \tagfor{for} loop that runs \tagfi{10} times
  \item create a \tagfor{for} loop inside that runs \tagfi{10} times
\end{enumerate}

To change the tagged sequence so that it is the same length as the input
sentence, we have to split tags that have contiguous words into separate tags
as well as have placeholder for non-tagged word. We are able to have tagged
sequence that is of the same length, like the below for the last line.

\hspace{20pt}
\tagfor{for} \tago{loop} \tago{that} \tago{runs} \tagfi{10} \tago{times}

The approach we are taking fits into a sequence-to-sequence modelling task
naturally. We will use a neural network approach to solve this task.
Specifically, we use a Recursive neural network that reads the sentences in one
word at a time and generates the tags after every word. Now the sentence
length and the tag sequence length are the same by artificially adding in the
dummy tag.


\section{Our Solution}
Recurrent neural networks has produced many impressive results on speech and
natural language processing as well computer vision task. These tasks are
characterized by an extremely large dimension space of the model. In our
specific task of sequence-to-sequence prediction, if we have $w$ words, and
$t$ different tag, then a sentence of length $n$ will have an input space
of $w^n$ and an output space of $t^n$. A large percentage of the input sentences
will be invalid because they do not make any sense in English as well as
invalid tag sequences in the output. However, this does not make this
task any less daunting given that there can be a complex combinations of words
that code can be expressed in speech.

Our approach is basically a supervised learning machine learning task. After
creating a corpus of correctly labelled sentences, we train the recurrent
neural network by feeding in each word in a sentence sequentially into the LSTM
and reading a tag at each time step. The process can be roughly abstracted as
shown in the figure:

% TODO : Figure

\subsection{Corpus}
% TODO : How many sentences
We have gathered a small corpus of labelled sentences that is used for
supervised training. Students, who are programmers, are given 4 different tasks
in which they must express their code by speaking directly to an experimenter.
The experimenter will listen to what the student said, and proceed to type on
the computer what is interpreted from his speech. The speech is generally a mixture
of movement commands as well as code definitions, such as variable declaration.
Throughout the entire experiment, the student is not able to directly type or
gesture to the experimenter his intent other than through his or her own voice.
% TODO : Reference Yixiu's part?

After each person has completed their task, their speech is then transcribed
and tagged manually. The sentences boundaries are inserted whenever the student's
speech reached an extended pause. This creates the realistic situation of
a real-time parsing of the spoken code. However, it also created many situations
where the spoken utterance did not contain any semantic elements like filler
words.

Since the programming task is known beforehand, we can easily infer the intent
of the spoken programme just by reading from the transcribed sentence. For example,
we differentiated between the word \texttt{number} in these 3 different context.

\begin{enumerate}
\itemsep0em
  \item \tago{go} \tago{to} \tago{line} \tago{number} \tago{7}
  \item \tagw{if}{if}{Blue} \tago{the} \tagvn{number} \tagfn{mod}
    \tagvv{2} \tago{is} \tago{even}
  \item \tagfn{printf} \tago{the} \tago{number} \tago{is} \tago{even}
\end{enumerate}

In the first sentence and in the last sentence, \texttt{number} is tagged
with the dummy tag. However, in the second case, the correct tag is
\texttt{variable-name}.

Even humans will have a difficult time tagging this, once the sentence are
taken out of context. We can easily compound the problem if all the 3 sentences
are combined into a single sentence. A student might indicate his wish to move
to the 7th line, followed by a \texttt{if} conditional and a \texttt{printf}.

It will be difficult for most sequence-to-sequence predictors to figure out
what the correct tags are without considering the surrounding word context and
sentence structure. Especially so when the same word is tagged differently. The
system must be robust to arbitrary combinations of the statements. However, the
size of the labelled corpus severely restricts the number of examples that the
model has seen and causes over-fitting to happen when the examples are sparse.

For example, given the complexity of the model parameters and small number of
variable names seen, the model might over-fit by remembering that some words
are very likely to be variable names compared to others. Words that are not
seen before have a high chance of being tagged with the dummy tag even though
they may occur with high probability with same tags like
\texttt{variable-names}. Hence the problem faced by a small corpus consists of
2 parts, one, the lack of sufficient examples in the structure of spoken code,
and two, the limited vocabulary of the corpus.  We shall discuss this problem
in further depth during the presentation of the results.

% TODO : Talk about how the generated corpus is useless
\subsubsection{Generated Corpus}
In the initial parts of the project, we set up to mitigate some of the issues
of corpus sparsity by generating an artificial corpus that contains an arbitrary
combination of certain code structures that we predict will appear in spoken code.

For example, we can easily create a corpus with a sentence prototype like so:

\hspace{20pt}
\tagif{if} \tagvn{i/j/k/...} \tagcond{less than/greater than} \tagvn{i/j/k/...}

This method helped to alleviate some of the sparsity problem that occurred in
the tagged transcribed speech. The corpus is now expanded with more possible
patterns of spoken code. However, this does not alleviate the problem of limited
vocabulary very much at all. Moreover, we see that the sentences in human spoken
code is punctuated with many dummy tags, but the generated corpus lacks such
features.

% TODO : Talk how about a modified version of dropout is used vs normal dropout
% TODO : Dropout by replacing with rare tag randomly
% TODO : Check ability to generalize
\subsubsection{Dropout}

\subsection{Word Embeddings}
First the words of the sentence are converted to word vectors via a word
embedding. The word embedding used in all our experiments is a float vector of
length 50. In our solution, the embedding vectors are randomly generated
initially before being updated using the back-propagated gradients during the
update step. A special \texttt{<unk>} symbol is given to words that we did not
encounter within the training corpus. We adopted word embeddings in our
training in the hope that similar words that have the same tags will be "close
together" in the word embedding space.

Here we also experimented with using pre-trained word embeddings from Wikipedia
that is trained using Glove \cite{pennington2014glove}.
% TODO : Differences in word embeddings
% TODO : Lemmatization

After converting the words in the sentence into word vectors, we then feed the
word vectors one at each time step into the LSTM.

% TODO : Fix ugly text
\begin{align*}
  &\text{for} &\text{loop } &\text{that } &\text{runs } &\text{10 } &\text{times } \\
  &\downarrow &\downarrow &\downarrow &\downarrow &\downarrow &\downarrow \\
  &[1.27, -0.2, 3, ...] &[0.4,...] &[-0.01, ...] &[-2, ...] &[4.2, ...] &[0.75, ...]
\end{align*}


\subsection{LSTM}
Long-short term memory (LSTM) refers to a specific architecture of Recursive
neural networks that is arranged as follows:

% TODO : LSTM Figure

$\sigma$ is the sigmoid function $f(x) = \frac{1}{(1 + e^x)}$, $\odot$ is the
matrix element-wise multiplication.

\begin{align*}
  s_t &= \lbrack h_{t-1}, x_{t} \rbrack \\
  f_t &= \sigma \left( W_f s_t + b_f \right) \\
  i_t &= \sigma \left( W_i s_t + b_i \right) \\
  c_t' &= tanh \left( W_c s_t + b_c \right) \\
  o_t &= \sigma \left( W_o s_t + b_o \right) \\
  c_t &= c_{t-1} \odot f_t + i_t \odot c_t' \\
  h_t &= o_t \odot tanh \left( c_t \right) \\
  y_t &= softmax \left( h_t W_s + b_s \right)
\end{align*}

At each word, the internal hidden states $c_t$ is updated as it advances from
one word to the next by removing irrelevant information


We added another layer of linear operation on the output state $h_t$ to get
$y_t$ before we are able to evaluate the softmax. This effectively frees the
LSTM to have the size of the hidden vector $c_t$ to be the same as the output
$y_t$. Adding another sigmoid operator on $y_t$ seems to have yielded extremely
bad training results that does not converge. This could be a case of bad
initialization of the output layer, but we did not have sufficient time to
fully test our hypothesis.
% TODO : Check output layer

The appeal of using a Recursive neural network like the LSTM over more
conventional methods like convolutional networks or fully connected neural
networks is that the input sequence can be made arbitrarily long, while still
retaining the capability of learning long range dependencies.


\subsubsection{Large Forget Gate Bias}
We initialized the forget gate bias $b_f$ of +2 instead of a zero vector after
considering the results from Jozefowicz, et al\cite{ICML2015JozefowiczZS}, show
that LSTMs become more capable at dealing with long range dependencies.

% TODO : Show better results of long range deps

\chapter{Related Work}
\section{SENNA}
The de facto Semantic Role labelling framework that provided a similar
neural network approach to the \cite{DBLP2011Collobert}

\section{Bi-directional LSTM-CNNs-CRF}


\section{Statistical Machine Translation}
There is an equally appealing framework for


\label{ch:related}


\chapter{Problem and Algorithm}
\section{Formal Description of Problem}
After defining the problem in terms of a sequence prediction, we now state the
loss function that is used as a measure of how likely the correct sequence of
tags are predicted given the input.

We measure the how well the model does by the negative log likelihood
of the labelled tagged sequence when given the conditional probability of all
tags for each word. This is the same loss function that will be optimized during
the learning of the model parameters $\theta$. The parameters refer to the
weight and bias matrices that are used the LSTM. Given a word $w \in W$ from
the list of words in our vocabulary, we obtain a probability of $P(T=t | w,
\theta)$ for every given tag that exists in our system.

\begin{equation}
  Loss(\theta) = -\sum_{w \in W} \sum_{t \in T} log(P(t | w, \theta)) I(t = \hat{t})
\end{equation}
where $\hat{t}$ is the correct tag for $w$ and $I$ is the 0-1 loss function

\subsection{Word Level Likelihood}

The loss function defined here is extremely similar to the one presented by
Collobert \cite{DBLP2011Collobert} under the section of Word-Level log
likelihood. This formulation just results in the correct path being having
increased likelihood but does not take into account of sentence level tag
dependencies that is captured in a more sophisticated sentence likelihood
function.

In brief, the sentence likelihood function considers a probability function
which places an explicit cost for certain tag-to-tag transitions. The extension
which was proposed in the paper involved learning the transition probability
and then producing a Sentence level log likelihood loss function. The best
sentence is optimal decoding over all possible paths that the tag sequence that
take using the Viterbi algorithm.

A more modern approach that is taken by the Xuezhe \cite{2016arXiv160301354M}
is to add a Conditional Random Field layer on top of the output of their Bi-directional
LSTM layer.


\subsection{Stochastic Gradient Descent}

The optimization of the function is done by employing gradient descent over the
parameters in the LSTM (including the word embedding vector).




\section{Design of Algorithm}



\chapter{Evaluation}
% TODO

\section{Implementation Details}
% TODO

We experimented with adding a L2 regularization term to the loss function but
% TODO put out results for using L2 reg
% TODO initialization divide by largest eigenvalue
% TODO large forget gate bias


\section{Experimental Setup}
% TODO

\section{Results}
% TODO Check L2 regularization differences
% TODO Difference with using word vectors

The model is extremely sensitive the words that are found within the corpus but generally
robust to noise that can result from mislabelling.



\chapter{Conclusion}
% TODO

\section{Contributions}
% TODO

\section{Future Work}


\subsection{Universal Code Speech}
Current work is done by transcribing spoken code from programmers that have a
good knowledge of the syntactical structures that are present in C. That causes
the transcribed speech have extremely clear structures that are reflective of
the syntax present in C code.

However, this creates problems in parsing when say we switch to a new
programming language like Python - The loop structure is distinctly different
in the sense that there are new keywords like range and enumerate. There is
also a more distinct lambda keyword in Python for defining anonymous functions.

The tagger will not be able to construct correct tags based on new words that
are not found within the training corpus. Having a good word embedding that
closely models keywords that perform the same functions may alleviate the problem
but it will be more informative to have an abstract representation of certain
coding concepts rather than rely on keywords and syntax to provide cues.


\subsection{Inter-Sentence Dependencies}
Certain contextual information can be brought across multiple sentences. This
situation happens most obviously when the variable or function names are used.
The context that a certain word is a particular variable type like an array is
lost, when the students use the same variable name over multiple sentences.

For example, in one the given task of the corpus gathering experiments, the
students have to create a string array and proceed to fill it with multiple
string values. It would be useful if the model allowed contextual information
that the array and its index is being used, rather than tag them independently
as \texttt{variable-name} and \texttt{variable-value}, we can tag the value as
the \texttt{variable-array-index}.


\printbibliography[title={Whole bibliography}]

\appendix
\chapter{Code}

\end{document}
